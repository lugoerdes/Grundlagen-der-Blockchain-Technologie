\documentclass[german]{../uebung}

\usepackage{../uebung-meta}
\usepackage{graphicx}
\usepackage[
	colorlinks=true,
	urlcolor=blue,
	linkcolor=green
]{hyperref}

\assignment{2}


\begin{document}

\begin{exercise}{1}
    \textbf{Vorgehensweise}
    Um die Aufgaben zu bearbeiten, betrachten wir zunächst die Definition aus der Vorlesung, um zu prüfen, ob die Anforderungen an eine Hashfunktion erfüllt sind:\\
    \begin{figure}[h]
        \centering
        \includegraphics*[scale=.5]{Definition Hashfunktion.png}
        \caption{Definition Hashfunktion}
    \end{figure}\\
    Im Anschluss betrachten wir die erweiterte Definition aus der Vorlesung für kryptografische Hashfunktionen, die aus folgenden Eigenschaften besteht:\\
    \begin{figure}[h]
        \centering
        \includegraphics*[scale=.5]{EinwegEigenschaft.png}
        \caption{Einweg-Eigenschaft}
    \end{figure}\\
    \begin{figure}[h]
        \centering
        \includegraphics*[scale=.5]{Kollisionsresistent.png}
        \caption{Kollisionsresistent Schwach}
    \end{figure}\\
    \begin{figure}[h]
        \centering
        \includegraphics*[scale=.5]{KollisionsresistentStark.png}
        \caption{Kollisionsresistent Stark}
    \end{figure}\\
    \textbf{a) \(h(x) = x \mod 7\)}\\
    \textbf{Komprimierung:}\\
    Ist erfüllt, da wir durch das Modulo 7, egal wie hoch die Eingabe ist, eine Zahl zwischen 0 und 6 erhalten werden.\\
    \textbf{Fast nie injektiv:}\\
    Ist erfüllt, die mögliche Definitionsmenge (=\(\infty\)) ist für diese Funktion als größer anzusehen, verglichen mit der Zielmenge (0,1,2,3,4,5,6).\\
    \textbf{Schnell berechenbar:}\\
    Ist erfüllt, da es sich beim Modulo um eine einfache Rechenoperation handelt, ist diese auch schnell berechenbar.\\
    Da wir gezeigt haben, dass es sich um eine Hashfunktion handelt, müssen wir nur noch die Kriterien für eine kryptografische Hashfunktion prüfen.\\
    \textbf{Einweg-Eigenschaft:}\\
    Ist nicht erfüllt, es ist möglich die Funktion umzukehren, um dann durch einfache Iteration zum x zu gelangen. Gehen wir von einem \(h(x)=3\) aus, dann können wir eine Gegenfunktion \(f(n)=3+n*7\) dazu bilden, die wir durchiterieren können. So kommen wir leicht auf das dazugehörige \(x\).\\
    \textbf{Kollisionsresitent Schwach:}\\
    Ist nicht erfüllt, zu einem gegebenen \(x\), lässt sich einfach ein \(y\) finden, zu dem gilt \(x \neq y\) und \(h(x)=h(y)\). Nehmen wir ein Beispiel. \(h(12)=12 \mod 7= 5\) finden wir viele beliebige \(x\), die zum selben Ergebnis führen. Durch folgende Funktion finden wir sehr einfach sehr viele \(x\) die zum gleichen Ergebnis führen: \(f(n)=5+n*7\).\\
    \textbf{Kollisionsresistent Stark:}\\
    Auch hier können wir eine einfache Kombination aus \(x\) und \(y\) finden, bei der gilt \(x \neq y\) und \(h(x)=h(y)\). Auch hier können wir beispielhaft die Funktion \(f(n)=5+n*7\) verwenden. Mithilfe der Funktion können wir beliebig viele x berechnen, sodass gilt \(h(x)=h(y)\) (beispielsweise \(x=12\) und \(y=19\)).\\
    \textbf{b) \(g(x) = x mod 12\)}\\
    Zunächst überprüfen wir, ob es sich bei der Funktion um eine Hashfunktion handelt. Dazu betrachten wir die oben beschriebenen Kriterien.\\
    \textbf{Komprimierung:}\\
    Ist erfüllt, da wir durch das Modulo 7, egal wie hoch die Eingabe ist eine Zahl zwischen 0 und 6 erhalten werden.\\
    \textbf{Fast nie injektiv:}\\
    Ist erfüllt, die mögliche Definitionsmenge (=\(\infty\)) ist für diese Funktion als größer anzusehen, verglichen mit der Zielmenge (Zahlen zwischen 0 und 11).\\
    \textbf{Schnell berechenbar:}\\
    Ist erfüllt, da es sich beim Modulo um eine einfache Rechenoperation handelt, ist diese auch schnell berechenbar.
    Da wir gezeigt haben, dass es sich um eine Hashfunktion handelt, müssen wir nur noch die Kriterien für eine kryptografische Hashfunktion prüfen.\\
    \textbf{Einweg-Eigenschaft:}\\
    Siehe Begründung in a).\\
    \textbf{Kollisionsresitent Schwach:}\\
    Siehe Begründung in a).\\
    \textbf{Kollisionsresistent Stark:}\\
    Siehe Begründung in a).\\
\end{exercise}\\

\begin{exercise}{2}
    Das Hashen mit benutzerspezifischen Informationen schützt besser gegen verschiedene Angriffe. Beispiele:
    Schutz vor Rainbow-Table-Angriffen: Bei Rainbow-Table-Angriffen werden vorgefertigte Tabellen genutzt, um Hashes effizient zu knacken. Durch die Verwendung führt dies zu individuelleren Hashes für jedes Passwort, insbesondere wenn Benutzer das gleiche Passwort verwenden. Der Angreifer müsste demnach zu jedem Hash eine neue Tabelle erstellen, was unpraktisch wäre.
    Erschweren von Brute-Force-Angriffen: Durch die Individualisierung erhöht sich gleichzeitig die Anzahl der möglichen Hashes.\\
\end{exercise}

\begin{exercise}{3}
    Alice und Bob wählen beide ihre Option. Beide können dann eine Hashfunktion \(h_j=h(Option_j)\). Da es lediglich drei Optionen gibt, wäre dies einfach durch Brute-Forcing zu knacken. Durch das Hinzufügen eines Salts vor dem hashen durch beispielsweise eine Zufallszahl kann dieses Problem gelöst werden. Nachdem sie ihre Wahl offengelegt haben, kann der andere Spieler sicherstellen, dass der ursprüngliche Zug nicht verändert wurde, indem er den bekanntgegebenen Zug hasht und mit dem zuvor erhaltenen Hash vergleicht.\\
\end{exercise}
\begin{exercise}{4}
    \textbf{a)}\\
    \textbf{b)}\\
    Da Computer A der schnellste ist gewinnt er, da er das Ziel von \(x=7852\) als erster erreicht.
    \textbf{c)}\\
    Um die Gewinnchance zu erhöhen, könnten die anderen Computer mithilfe einer anderen Strategie ihre Gewinnchance erhöhen.
    \textbf{d)}\\
\end{exercise}
\begin{exercise}{5}

\end{exercise}

\end{document}
