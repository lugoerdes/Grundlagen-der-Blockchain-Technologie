\documentclass[german]{../uebung}

\usepackage{../uebung-meta}
\usepackage{graphicx}
\usepackage[
	colorlinks=true,
	urlcolor=blue,
	linkcolor=green
]{hyperref}

\assignment{2}


\begin{document}

\begin{exercise}{1}
	\textbf{Vorgehensweise}
	Um die Aufgaben zu bearbeiten, betrachten wir zunächst die Definition aus der Vorlesung, um zu prüfen, ob die Anforderungen an eine Hashfunktion erfüllt sind:\\
	\begin{figure}[h]
		\centering
		\includegraphics*[scale=.5]{Definition Hashfunktion.png}
		\caption{Definition Hashfunktion}
	\end{figure}\\
	Im Anschluss betrachten wir die erweiterte Definition aus der Vorlesung für kryptografische Hashfunktionen, die aus folgenden Eigenschaften besteht:\\
	\begin{figure}[h]
		\centering
		\includegraphics*[scale=.5]{EinwegEigenschaft.png}
		\caption{Einweg-Eigenschaft}
	\end{figure}\\
	\begin{figure}[h]
		\centering
		\includegraphics*[scale=.5]{Kollisionsresistent.png}
		\caption{Kollisionsresistent schwach}
	\end{figure}\\
	\begin{figure}[h]
		\centering
		\includegraphics*[scale=.5]{KollisionsresistentStark.png}
		\caption{Kollisionsresistent Stark}
	\end{figure}\\
	\textbf{a) \(h(x) = x \mod 7\)}\\
	\textbf{Komprimierung:}
	Ist erfüllt, da wir durch das Modulo 7, egal wie hoch die Eingabe ist, eine Zahl zwischen 0 und 6 erhalten werden.\\
	\textbf{Fast nie injektiv:}
	Ist erfüllt, die mögliche Definitionsmenge (=\(\infty\)) ist für diese Funktion als größer anzusehen, verglichen mit der Zielmenge (0,1,2,3,4,5,6).\\
	\textbf{Schnell berechenbar:}
	Da es sich beim Modulo um eine einfache Rechenoperation handelt, ist diese auch schnell berechenbar.\\
	\textbf{b) \[g(x) = x mod 12\]}\\
	\textbf{Komprimierung:}
	Ist erfüllt, da wir durch das Modulo 7, egal wie hoch die Eingabe ist eine Zahl zwischen 0 und 6 erhalten werden.
	\textbf{Fast nie injektiv:}
	Ist erfüllt, die mögliche Definitionsmenge (=\(\infty\)) ist für diese Funktion als größer anzusehen, verglichen mit der Zielmenge (Zahlen zwischen 0 und 11).
	\textbf{Schnell berechenbar:}
	Da es sich beim Modulo um eine einfache Rechenoperation handelt, ist diese auch schnell berechenbar.
\end{exercise}

\begin{exercise}{2}
\end{exercise}

\begin{exercise}{3}

\end{exercise}
\begin{exercise}{4}
	\textbf{a)}
	\textbf{b)}
	\textbf{c)}
	\textbf{d)}
\end{exercise}
\begin{exercise}{5}

\end{exercise}

\end{document}
