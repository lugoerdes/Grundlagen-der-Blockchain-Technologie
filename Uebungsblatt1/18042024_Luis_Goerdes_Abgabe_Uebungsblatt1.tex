\documentclass[german]{../uebung}

\usepackage{../uebung-meta}
\usepackage{graphicx}
\usepackage[
	colorlinks=true,
	urlcolor=blue,
	linkcolor=green
]{hyperref}

%%%%%%%%%%%%%%%%%%%%%%%%%%%%%%%%%%%%%%%%%%%%%%%%%%%%%%%%%%%%%%%%%%%%%%%%%%%%
% README
%%%%%%%%%%%%%%%%%%%%%%%%%%%%%%%%%%%%%%%%%%%%%%%%%%%%%%%%%%%%%%%%%%%%%%%%%%%%

% How to use this:
% 1. Add your data to uebung-meta.sty (you only need to do this once)
% 2. Copy this file and name it something useful
% 3. Set the assignment to the right value
% 4. Use the exercise enviroment to separate your solutions for the different exercises from each other.

% DO NOT CHANGE uebung.cls! If you need more packages just add a new \usepackage somewhere in this file before the \begin{document}

% The environment "exercise" takes one parameter (the exercise number). 
% This way you can skip exercises if you like. Example:
% 
% \assignment{3}
% \begin{exercise}{8}
% ...
% \end{exercise}
% 
% The solution to exercise 3.8 (3rd assignment, 8th exercise) goes where 
% the dots are.

% If the total page number shows up as ?? in the footer you need to compile a second time.

% There are some predefined command you can use:
% \O for Big-O notation
% \scan and \sort for scan and sort complexity

% Which assignment is this?
\assignment{Nr}


\begin{document}

\begin{exercise}{1.1}
	Die Signatur wurde mittels Thunderbird erstellt und über folgende E-Mail-Adresse versendet: lugoerdes@gmail.com
	\begin{figure}[h]
		\centering
		\includegraphics*[scale=.5]{Uebungsblatt1.png}
		\caption{Gesendete Mail}
	\end{figure}
\end{exercise}

\begin{exercise}{1.2}
	Das Halving ist ein Ereignis in der Blockchain-Technologie vor allem bekannt durch den Bitcoin, das beim Bitcoin ungefähr alle vier Jahre stattfindet, oder genauer gesagt, alle 210.000 Blöcke (Bei anderen Coins kann die Anzahl der Blöcke abweichen; Beispiel Litecoin bei 840.000 Blöcken). Während des Halvings wird die Belohnung, die Miner für das Hinzufügen eines neuen Blocks zur Blockchain erhalten, halbiert. Ursprünglich erhielten Miner 50 Bitcoins pro Block; nach dem ersten Halving fiel diese Belohnung auf 25 Bitcoins, dann auf 12,5 Bitcoins und weiter so fort.\\
	Auch wenn es nie genau von den Entwicklern spezifiert wurde, könnten folgende Gründe für den Mechanismus gelten:\\
	Kontrolle der Inflation: Bitcoin hat eine maximale Grenze von 21 Millionen Coins, die jemals existieren werden. Das Halving hilft, die Rate, mit der neue Bitcoins erschaffen werden, zu verlangsamen, was die Inflation vermindert und den Wert der Währung über die Zeit hinweg erhält.\\
	Verlängerung der Lebensdauer der Blockbelohnungen: Indem die Belohnungen im Laufe der Zeit verringert werden, wird sichergestellt, dass die Erzeugung neuer Bitcoins nicht zu schnell geschieht. Dies verlängert die Periode, in der Mining-Aktivitäten profitabel sein können, trotz sinkender Belohnungen.\\
	Förderung des langfristigen Engagements: Das Halving ermutigt Miner, in die Sicherheit und Infrastruktur des Netzwerks zu investieren und langfristig beteiligt zu bleiben, da die Belohnungen zwar geringer werden, der Wert eines einzelnen Bitcoins jedoch potenziell steigen kann.\\
	Das Halving bei 503 Blöcken bedeutet, dass nur noch 503 Blöcke gemined werden müssen, bis das nächste Halving eintritt und die Blockbelohnung erneut halbiert wird. Dies ist ein signifikantes Ereignis für Investoren und Miner, da es sowohl Auswirkungen auf die Rentabilität des Minings als auch auf den Marktpreis von Bitcoin haben kann. Diese Auswirkungen werden sich aber erst in der Zukunft zeigen. Bei den letzten Halvings ist der Bitcoinpreis über ein paar weniger Monate deutlich angestiegen.\\
	Zum Zeitpunkt der Bearbeitung des Arbeitsblattes hat das Halving-Event für den Bitcoin bereits stattgefunden und hat nur zu einer kleinen Erhöhung des Bitcoinpreises geführt.
\end{exercise}

\end{document}
